The front end of the project was crafted using a combination of HTML, CSS, and JavaScript, which outputted an interactive user interface. HTML was utilized to structure the content, creating the skeleton of the web pages. CSS was used in styling this content and JavaScript was integral for form validations and fetching content from our services.

To bring this front-end framework to life, the application was deployed on a Flask server, utilizing Python to manage the backend operations. Flask, a lightweight and flexible web framework, allowed for the efficient handling of requests and responses, serving the HTML, CSS, and JavaScript files to our orchestration team to implement. Python's integration with Flask facilitated the creation of routes and endpoints, enabling our backend services to interact with the front end. The use of Flask and Python in the backend complemented the front-end technologies, resulting in a cohesive and well-integrated web application.