In this section we will discuss the implmentation of defense or our group project in detail. For our groups approach to defense we used a two part system built of a honeypot and a script to control traffic. The goal of the honeypot is to provide a avenue for traffic that deviates from the norm. Once this deviated or unauthorized traffic is found the script uses iptables to block incoming traffic effectivley removing traffic from unauthorized users, such as those who would scan the server or try to connect to a fake service on the honey pot. This defensive approach assumes authorized users will know what is in scope and not try to connect to or scan the honeypot. 

The honeypot used for defense for this project is opencanary. Opencanary was choosen beacause of its familiarity due to classes as well as its simplicity and usability with the rest of the groups implementation. The usability stems from opencanary being built with docker and allows for our group to seamlessly integrate opencanary with our other dockerfiles allowing for the configuration to be quick and, once the configuration file is set building and running the dockerfile easy. The easy deployment of this honeypot will work perfectly for the expected attacks further into the project and will allow deployment of the honeypot on a new server to be easily contained in a script that calls the build and run command. 
