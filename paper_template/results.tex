After combining all components into a final working network with a web app with multiple services, the next stage was to analyze each of the services. 
Analyzing the services (otherwise known as enumeration) is the first step that a malicious actor might take. Using surveillance tools 
such as nmap, gobuster, or a ping sweep to map the hops of the system to checking what ports are available. One thing to note is that the tools listed 
are not exclusively used by malicious actors but are used by admins to provide better Cybersecurity solutions which is what the researchers of this 
study are applying.

We utilize nmap, a command line tool to determine what ports and services are currently on offer on a system,   
with the \verb|-sC| and \verb|-sV| flags which, respectively, will gather detailed information and the version that a service is operating on the network. In the 
study, the services are revealed as well as what ports those services operate on. For example it can be observed that there is an OpenSSH service 
operating on port 22 using tcp. A Defensive measure to protect this service would be to look at the \hyperlink{https://cve.mitre.org/}{CVE} (Common 
Vulnerabilities and Exposures) for OpenSSH. Looking at the CVE for OpenSSH, if the service is the most updated version, then admins might try to defend the encrypted 
key protocol to prevent messages from being deleted.

The ports that were revealed through an nmap scan also show that there is little security in protecting routing. This was observed as typing the port 
number at the end of the url such as “:9000” would reveal a webpage that had a scramble of letters and characters. This vulnerability could be remedied 
through proper routing controls so that url redirection attacks are minimized. In addition to eliminating security threats, testing was implemented on 
SSH for weak passwords. Passwords that were identified as weak were changed.

The iptable rules proved relatively effective in minimizing what could be discovered on the server. While they are found to be open with an nmap scan, they do drop all traffic that attempts to connect to them. By adding this element to the server we provide more opportunity for distraction and make it significantly more difficult to find something of value to an attacker. 

After testing our logging we found that former logs needed to be purged to make traffic from the expected attack clearer so we did so. After running the Nmap scan to generate traffic to test as referenced in the Evaluation section \ref{Evaluation} we found that our logs had produced multiple logs which when analyzed had traffic meaning logging was working as intended.


