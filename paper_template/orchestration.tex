The orchestration for this deployment concerns itself with two key areas. The first area, services, 
have been designed utilizing docker containers and docker compose for simple build up and tear down. 
Our docker compose solution utilizes two docker images, one running NGINX as a reverse proxy, 
and the other running a flask server locally which serves as an endpoint for the NGINX server. 

The second area, defense, utilizes OpenCanary and IPTables. While OpenCanary is also a docker container, the 
complexity in building it warrants its own separate orchestration. To accomplish this, we utilize 
some basic bash scripting to automate the necessary commands to run OpenCanary with the configuration and 
services required. We also include IPTables commands part of this bash script. These rules filter ports and 
drop traffic that attempts to enumerate OpenCanary and its ports. 

By automating most of the more complex tasks, our services and defenses can be deployed quickly 
and by a greater number of individuals, even those not intimately familiar with its inner workings. 
The use of docker containers and docker compose also allow our deployment to be mobile, only requiring 
prospective users to install a small number of programs (i.e., docker, docker compose, git). 


