	In a basic connection between a client and a host, one computer would send a request that would get relayed by a server to the destination computer. In a proxy environment, the proxy sits in the middle between the client and the server, relaying the requests to the server on the client's behalf. One defense implementation used in this project is a reverse proxy. Unlike a forwarding proxy, which forwards network traffic on your behalf, a reverse proxy forwards traffic to a specific destination, usually you. They are positioned in front of web servers and forward browser requests to those servers. This is typically done for a few reasons, including load balancing, cyber defense, caching, and even SSL encryption\cite{CloudFlare_2024j}. A load balancer is used to distribute traffic among several servers, preventing direct traffic from overloading any given one, like a mail room. In terms of defense, it obfuscates a client's actual IP address, making it a more difficult target; and ideally, the reverse proxy is a third-party that has the infrastructure to defend against attacks. 

	For this project, rather than use a third-party company to act as a reverse proxy, a software tool called NGINX was employed. NGINX is a free, open-sourced web server released in 2004 that can be used as either an HTTP, reverse proxy, mail proxy, and generic TCP/UDP proxy server. It is popular and dependable enough to be used by major companies like Netflix and Dropbox. In fact, it is one of the top four most used services, and incredibly popular in the Docker community as the most commonly deployed technology in docker containers\cite{wikiNginx_2024m}. For this reason, it made it an easy choice when looking for a technology to act as the project's reverse proxy that could be easily incorporated into our Docker implementation. 

	Honeypots are widely used tools that can provide invaluable data when in a defensive position. In general, they’re decoys that lure attackers in to allow a defensive team to study the incoming attacks. In this project specifically, \hyperref{https://opencanary.readthedocs.io/en/latest/}{opencanary} has been implemented. Major benefits of this implementation are its extremely low resource requirements and the easy extensibility. It can be a good way to get IPs scanning networks, as the only case it should be discovered is if someone is scanning the entire network searching for vulnerabilities. Use of this stemmed from a want to diversify the implemented defensive techniques. By incorporating OpenCanary it has allowed greater surface from which logs can be collected which is extremely valuable for an evolving defense. Additionally, malicious IPs can be identified more regularly, presenting the opportunity to create IP blocking rules in the future. 
